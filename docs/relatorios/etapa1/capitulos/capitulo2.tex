% $Id$
% ---------------------------------------------------------------------------
%
%  This is part of the SIGI.
%  Copyright (C) 2008 Interlegis
%  See the file relatorio.tex for copying conditions.
%

\section{Requisitos}
\label{sec:requisitos}

\subsection{Detalhamento do Produto}
O produto a ser desenvolvido deverá contemplar as seguintes
funcionalidades:

\begin{itemize}
\item Gerenciamento das informações básicas de identificação das Casas
  Legislativas;
\item Gerenciamento das informações sobre partidos políticos,
  parlamentares, seus mandatos, e legislaturas;
\item Gerenciamento de sessões legislativas, mesas diretora e sua
  composição;
\item Gerenciamento de convênios entre o Interlegis e as Casas
  Legislativas;
\item Gerenciamento dos equipamentos e serviços disponibilizados para
  as Casas Legislativas;
\item Conjunto de autorização e de configuração de perfis de acesso ao
  sistema;
\item Permitir que o usuário efetue pesquisas no banco de dados
  utilizando filtros de seleção;
\item ``Rollback'' (retorno) das alterações, em tempo de
  edição dos dados, nos casos de falhas;
\item Geração de relatórios em PDF, dos dados estratégicos e
  de resultados de pesquisas;
\item Manutenção do histórico das atividades com as Casas Legislativas.
\end{itemize}

Considerar as seguintes características:

\begin{itemize}
\item Interface web compatível com os padrões e recomendações da World
  Wide Web Consortium (W3C);
\item Uso de \emph{AJAX} para validação de dados e para otimização e
  facilidade de uso do sistema;
\item Uso de um \emph{Sistema de Controle de Versões} para gerenciamento e
  acompanhamento da codificação;
\item Ajuda \textit{online} e contextual do sistema;
\item Instalação simplificada;
\item Sistema multi-usuário em rede, concebido para operar em três
  camadas distintas: apresentação (navegador web), aplicação e dados.
\end{itemize}

\subsection{Requisitos de Hardware}
\begin{itemize}
\item Disco: 100 MB ou superior;
\item Memória RAM: 256 MB ou superior;
\item Processador: Pentium III ou superior;
\end{itemize}

\subsection{Requisitos de Software}
\begin{itemize}
\item Sistemas operacionais GNU/Linux ou de família Unix BSD;
\item Servidor web: Apache 2 ou superior;
\item SGBD: PostgreSQL 8.3 ou superior;
\item Linguagem/Interpretador: Python 2.5 ou superior (< 3.0);
\item Framework: Django 0.97-pre (trunk) ou 1.0.
\end{itemize}

\subsection{Requisitos de Performance}
\begin{itemize}
\item Considerar o uso do sistema operacional GNU/Linux ou FreeBSD;
\item Uso do servidor web Apache para a aplicação;
\item Uso do servidor Lighttpd para os arquivos estáticos;
\item Cache das transações com o Banco de Dados e da saída
  (\textit{output}) das requisições HTTP.
\end{itemize}

\subsection{Requisitos de Documentação}
\begin{itemize}
\item Documentação da \textit{Application Programming Interface} (API)
  do software;
\item Manual de configuração e de implantação do sistema;
\item Manual do usuário;
\item Ajuda \textit{online} e contextual do sistema;
\item Arquivos LEIA-ME e de licenciamento.
\end{itemize}

\subsection{Licenças}
O novo SIGI utiliza \emph{softwares livres}, os quais possuem um
licenciamento sem restrições de uso.

Segue abaixo uma lista dos principais \emph{softwares} necessários
para o desenvolvimento do SIGI e suas respectivas licenças:

\begin{description}
\item[Apache:] Apache License 2.0
\item[PostgreSQL:] BSD License
\item[Python:] Python License
\item[Django:] BSD License
\end{description}

O SIGI será publicado como \emph{software livre} utilizando a licença
\textit{GNU General Public License (GNU GPL)} em sua versão 3 e estará
disponível no repositório de \emph{software livre}
Colab\footnote{Endereço web do Colab: http://colab.interlegis.gov.br}
do Interlegis.
%
% Local variables:
%   mode: flyspell
%   TeX-master: "relatorio.tex"
% End:
%
