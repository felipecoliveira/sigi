% $Id$
% ---------------------------------------------------------------------------
%
%  This is part of the SIGI.
%  Copyright (C) 2008 Interlegis
%  See the file relatorio.tex for copying conditions.
%

\section{Requisitos}
\label{sec:requisitos}

\subsection{Detalhamento do Produto}
O produto a ser desenvolvido deverá contemplar as seguintes
funcionalidades:

\begin{itemize}
\item Gerenciamento das informações das Casas Legislativas;
\item Gerenciamento de partidos políticos, parlamentares e seus
  mandatos, e de legislaturas;
\item Gerenciamento de sessões legislativas, mesas diretora e de sua
  composição partidária;
\item Gerenciamento de convênios das Casas Legislativas junto ao
  Interlegis;
\item Gerenciamento dos equipamentos fornecidos e serviços prestados
  às Casas Legislativas;
\item Conjunto de autorização e configuração de perfis de acesso ao
  sistema;
\item Permitir que o usuário efetue pesquisas no banco de dados
  utilizando filtros de seleção;
\item Permitir o ``rollback'' (recuo) das alterações em tempo de
  edição dos dados;
\item Permitir a geração de relatórios em PDF dos dados estratégicos e
  de pesquisas;
\item Manter o histórico das atividades com as Casas Legislativas.
\end{itemize}

Deverá também ser considerado as seguintes características:

\begin{itemize}
\item Interface web compatível com os padrões e recomendações da World
  Wide Web Consortium (W3C);
\item Uso de \emph{AJAX} para validação de dados e para otimização e
  facilidade de uso do sistema;
\item Uso de um \emph{Sistema de Controle de Versões} para gerenciamento e
  acompanhamento da codificação;
\item Ajuda \textit{online} e contextual do sistema;
\item Instalação simplificada;
\item Sistema multi-usuário em rede concebido para a operacionalização
  em três camadas distintas: apresentação (navegador web), aplicação e
  dados.
\end{itemize}

\subsection{Requisitos de Hardware}
Requisitos de \textit{hardware}necessários para que a aplicação
funcione corretamente:

\begin{itemize}
\item Aproximadamente 100 MB de espaço em disco;
\item Mínimo de 256 MB de memória RAM;
\item Processador compatível ao Pentium III ou superior;
\end{itemize}

\subsection{Requisitos de Software}
Requisitos de \textit{software} necessários para que a aplicação
funcione corretamente:

\begin{itemize}
\item Sistemas operacionais GNU/Linux ou de família Unix BSD;
\item Apache 2 ou superior;
\item PostgreSQL 8.3 ou superior;
\item Python 2.5 ou superior (< 3.0);
\item Django 0.97-pre (trunk) ou 1.0.
\end{itemize}

\subsection{Requisitos de Performance}
Requisitos necessários para tirar o melhor proveito da aplicação:

\begin{itemize}
\item Considerar o uso do sistema operacional GNU/Linux ou FreeBSD;
\item Uso do servidor web Apache para a aplicação;
\item Uso do servidor Lighttpd para os arquivos estáticos;
\item Cache das transações com o Banco de Dados e da saída
  (\textit{output}) das requisições HTTP.
\end{itemize}

\subsection{Requisitos de Documentação}
Documentação necessária que deverá vir a ser elaborada para servir de
suporte ao produto:

\begin{itemize}
\item Documentação da \textit{Application Programming Interface} (API)
  do software;
\item Manual de configuração e implantação do sistema;
\item Manual do usuário;
\item Ajuda \textit{online} e contextual do sistema;
\item Arquivos LEIA-ME e de licenciamento.
\end{itemize}

\subsection{Licenças}
O novo SIGI se baseia apenas em aplicações \emph{livres}, as quais
possuem um licenciamento sem nenhuma restrição de uso. A obtenção das
mesmas também são gratuitas.

Segue abaixo uma lista dos principais \emph{softwares} necessários
para o desenvolvimento do SIGI e suas respectivas licenças:

\begin{description}
\item[Apache:] Apache License 2.0
\item[PostgreSQL:] BSD License
\item[Python:] Python License
\item[Django:] BSD License
\end{description}

O SIGI será publicado como \emph{software livre} utilizando a licença
\textit{GNU General Public License (GNU GPL)} em sua versão 3, assim
como os outros produtos do Interlegis, e estará disponível no
repositório de \emph{software livre} Colab\footnote{Endereço web do
  Colab: http://colab.interlegis.gov.br} do Interlegis.
%
% Local variables:
%   mode: flyspell
%   TeX-master: "relatorio.tex"
% End:
%
