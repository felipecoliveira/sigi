% $Id$
% ---------------------------------------------------------------------------
%
%  This is part of the SIGI.
%  Copyright (C) 2008 Interlegis
%  See the file relatorio.tex for copying conditions.
%

\section{Visão Geral}
\label{sec:visaogeral}
O atual Sistema de Informações Gerenciais do Interlegis, ou SIGI,
fornece um conjunto limitado de informações sobre as Casas
Legislativas e não contempla a atual realidade do \emph{Programa
  Interlegis}.

O sistema é executado em ambiente \emph{Microsoft Windows} utilizando
a plataforma \emph{Microsoft Access}, o qual possui uma série de
limitações tecnológicas, de manutenibilidade, e de acesso às
informações pelos usuários, bem como também o acesso através de outros
\textit{softwares}, gerando, neste caso, redundância e disparidade de
dados.

Foi observado por parte dos usuários a necessidade de um sistema com
interface \textit{web} e com uma \emph{base de dados} única e
centralizada, tornando o acesso flexível e as informações atualizadas
refletindo diretamente nos outros sistemas do Interlegis.

O novo sistema será desenvolvido utilizando a plataforma \emph{Python}
com o framework de desenvolvimento \emph{Django}, o qual cumpre os
requisitos necessários para a implementação do sistema. Além do mais,
o Django permitirá uma evolução natural do sistema e grande facilidade
na manutenção do mesmo.

O sistema, inicialmente, será dividido em 7 aplicações (ou
componentes) Django:

\begin{enumerate}
\item \verb|sigi.apps.casas| (Casas Legislativas)
\item \verb|sigi.apps.contatos| (Contatos)
\item \verb|sigi.apps.convenios| (Convênios)
\item \verb|sigi.apps.inventario| (Inventário)
\item \verb|sigi.apps.mesas| (Mesas Diretoras)
\item \verb|sigi.apps.parlamentares| (Parlamentares)
\item \verb|sigi.apps.servicos| (Serviços)
\end{enumerate}

Esta ``componentização'' permitirá o reaproveitamento dessas mesmas
aplicações para outros projetos futuros do Interlegis baseado em
Django.

\subsection{Descrição das Atividades}
Esta seção descreve as principais etapas de desenvolvimento do sistema
a serem cumpridas. Desde o planejamento e levantamento de requisitos
até a implantação do sistema e transferência de tecnologia à equipe do
Interlegis.

\subsubsection{Primeira Etapa}
\begin{enumerate}
\item Levantamento de requisitos (Seção \ref{sec:requisitos});
\item Identificação dos \emph{Casos de Uso} (Seção \ref{sec:casos});
\item Definição do \emph{Modelo de Dados} (Seção \ref{sec:modelo});
\item Esquema da Base de Dados (Seção \ref{sec:esquema});
\item Protótipo de interface gráfica (Seção \ref{sec:prototipo}).
\end{enumerate}

\subsubsection{Segunda Etapa}
\begin{enumerate}
\item Codificação do sistema;
\item Transformação do protótipo em interface para o sistema;
\item Implantação de sistema \textit{alpha} em um servidor de testes;
\item Avaliação final do sistema.
\end{enumerate}

\subsubsection{Terceira Etapa}
\begin{enumerate}
\item Documentação do sistema desenvolvido;
\item Documentação dos parâmetros de configuração do sistema;
\item Manual de implantação (\textit{deployment}).
\end{enumerate}

\subsubsection{Quarta Etapa}
\begin{enumerate}
\item Manual de usuário;
\item Treinamento;
\item Implantação do sistema em servidor em produção;
\item Transferência de tecnologia para equipe do Interlegis;
\item Entrega final do produto.
\end{enumerate}

%
% Local variables:
%   mode: flyspell
%   TeX-master: "relatorio.tex"
% End:
%
